\section{Vocabulaire}

\begin{frame}{Clavier et keymap}
  \begin{itemize}
    \item Clavier~: dispositif \emph{physique} permettant la saisie. Exemple~:
      les différents modèles du PIE, le TypeMatrix. \pause

    \item Keymap~: utilisation \emph{logique} du clavier par le
      système. Exemple~: les dispositions QWERTY, QWERTZ, AZERTY, Dvorak US,
      Bépo.
  \end{itemize}
\end{frame}



\subsection{Clavier}

\begin{frame}{Anatomie d’un clavier}
  \begin{itemize}
    \item Ensemble de touches disposées sur une «~planche~» \pause

    \item Plusieurs groupes~: pavé alphanumérique, directionnel, numérique,
      touches de fonction, … \pause

    \item Plusieurs dispositions écrites (labels sur les touches) \pause

    \item Plusieurs paramètres ergonomiques~: taille, éloignement, disposition,
      profondeur, résistance, seul d’activation des touches
  \end{itemize}
\end{frame}

\begin{frame}{Les claviers aujourd’hui}
  \begin{itemize}
    \item Pavé alphanumérique, directionnel, numérique, touches de fonctions
      (parfois touches multimédia) \pause

    \item Rangées du pavé alphanumérique décalées en escalier («~clavier
      droit~») \pause

    \item Touches «~Backspace~» et «~Enter~» peu accessibles \pause
  \end{itemize}

  Il existe des alternatives (rarement utilisées)~: \pause
  \begin{itemize}
    \item Claviers compacts~: suppression de pavés (souvent le pavé numérique)
      \pause

    \item Touches en matrice \pause

    \item «~Backspace~» et «~Enter~» au milieu du pavé alphanumérique
  \end{itemize}
\end{frame}



\subsection{Keymap}

\begin{frame}{Caractéristiques d’une keymap}
  On évalue une keymap par rapport au langage saisi (français, anglais, Python,
  C++, …) \pause

  Quelques critères~:
  \begin{itemize}
    \item Éloignement des touches par rapport à la fréquence d’utilisation
      \pause

    \item Répartition lors de séquences fréquentes \pause

    \item Touches mortes~: agissent a posteriori
  \end{itemize}
\end{frame}

\begin{frame}{Les keymaps aujourd’hui}
  \begin{itemize}
    \item Censées être spécialisées pour chaque langue. \pause

    \item De trop nombreuses dispositions existent~: colemak, svorak, dvorak
      (gaucher, droiter, international), bepo, bepo-fr, qwerty, azerty, …
  \end{itemize}
\end{frame}
